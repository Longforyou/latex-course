\usepackage{amsfonts}
\usepackage{amsmath}
\usepackage{amssymb}
\usepackage{mathptmx}
\usepackage{color}


\newcommand{\bftt}[1]{\textbf{\texttt{#1}}}
\newcommand{\comment}[1]{{\color[HTML]{008080}\textit{\textbf{\texttt{#1}}}}}
\newcommand{\cmd}[1]{{\color[HTML]{008000}\bftt{#1}}}
\newcommand{\bs}{\char`\\}
\newcommand{\cmdbs}[1]{\cmd{\bs#1}}
\newcommand{\lcb}{\char '173}
\newcommand{\rcb}{\char '175}
\newcommand{\cmdbegin}[1]{\cmdbs{begin\lcb}\bftt{#1}\cmd{\rcb}}
\newcommand{\cmdend}[1]{\cmdbs{end\lcb}\bftt{#1}\cmd{\rcb}}



% ******************************** Meta-data ***********************************
\mode<presentation>
{
  \usetheme{Madrid}
  \setbeamercovered{transparent}
}


\usepackage{caption}
\captionsetup{font=scriptsize, labelfont=scriptsize, justification=centering}

\title{Writing papers and thesis using \LaTeX2e}

\author {Krishna Kumar \inst{*}\thanks{kks32@cam.ac.uk} }

\institute[ University of Cambridge ] % (optional, but mostly needed)
{
  \inst{1}%
  King's College\\
  University of Cambridge
}

\date[LaTeX course 2014] % (optional, should be abbreviation of conference name)
{King's Computing Workshop, January 2014}

\pgfdeclareimage[height=0.2cm]{university-logo}{figs/cambridge.png}


% Delete this, if you do not want the table of contents to pop up at
% the beginning of each subsection:
%\AtBeginSubsection[]
%{
%  \begin{frame}<beamer>{Outline}
%    \tableofcontents[currentsection,currentsubsection]
%  \end{frame}
%}


% If you wish to uncover everything in a step-wise fashion, uncomment
% the following command: 

% \beamerdefaultoverlayspecification{<+->}
