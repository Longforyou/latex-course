\documentclass[times]{article}
\usepackage{amsfonts}
\usepackage{amsmath}
\usepackage{amssymb}
\usepackage{mathptmx}
\usepackage{cleveref}

\title{Article with lots of equations}
\author{Krishna Kumar}
\date{}

\begin{document}

\maketitle

You can also do an inline equation of $(a+b)^2 = a^2 + b^2 + 2ab$.

Another inline equation is the Euler's equation: $e^{i\pi}=-1$. This beautiful equation connects three major constants of mathematics, Euler's Number \textit{e}, the ratio of the circumference of a circle to its diameter, pi, and the square root of -1, i.e., \textit{i}.


The Schr\"{o}dinger's~\cref{eq:schrodinger}. Also the one with multiple 
equations and a single number is~\cref{eq:yequation}.

\begin{equation}
i \hbar \frac{\partial}{\partial t} \Psi(r,t) = 
\left[\frac{-\hbar^2}{2\mu}\nabla^2+V(r,t)\right]\Psi(r,t)
\label{eq:schrodinger}
\end{equation}

% Try without the equation number
\begin{equation*}
E^2 = (pc)^2 + (m_0 c^2)^2
\end{equation*}

%5. It's time to have multiple equations and align them. Note: & aligns the 
% equations and see use of \nonumber. Uncomment the following section and see 
% how it affects the output.  try using [fleqn] as a document class option to 
%see what happens  then add & on left hand side of the equals sign on all 
%equations and see the output like y & = ax+ b

\begin{align}
	y   & =  ax+b \nonumber\\
	y+1 & = ax+(b+1)\\
	    & = ax+(b+2)-1
\end{align}


\begin{align}
\label{eq:yequation}
\begin{aligned}
	y   & =  ax+b \\
	y+1 & = ax+(b+1)\\
	    & = ax+(b+2)-1
\end{aligned}
\end{align}


% Try using gather environment to center the equations

\begin{gather}
	y     =  ax+b \nonumber\\
	y+1   =  ax+(b+1)\\
	      =  ax+(b+2)-1
\end{gather}
\end{document}
