%\documentclass[times]{article}
%\usepackage{amsfonts}
%\usepackage{amsmath}
%\usepackage{amssymb}
%\usepackage{mathptmx}
%\usepackage{cleveref}
%\title{Maths}
%\author{Krishna Kumar}
%\date{}
\begin{document}
%\maketitle

You can also do an inline equation of (a+b)2 = a2 + b2 + 2ab

The Schr\"{o}dinger's schrodinger. Also the one with multiple 
equations and a single number is

%5. It's time to have multiple equations and align them. Note: & aligns the 
% equations and see use of \nonumber. Uncomment the following section and see 
% how it affects the output.  try using [fleqn] as a document class option to 
%see what happens  then add & on left hand side of the equals sign on all 
%equations and see the output like y & = ax+ b

%\begin{align}
%	y   & =  ax+b \nonumber\\
%	y+1 & = ax+(b+1)\\
%	    & = ax+(b+2)-1
%\end{align}
%
%
%\begin{align}
%\label{eq:yequation}
%\begin{aligned}
%	y   & =  ax+b \\
%	y+1 & = ax+(b+1)\\
%	    & = ax+(b+2)-1
%\end{aligned}
%\end{align}
%
%
%% Try using gather environment to center the equations
%
%\begin{gather}
%	y     =  ax+b \nonumber\\
%	y+1   =  ax+(b+1)\\
%	      =  ax+(b+2)-1
%\end{gather}
\end{document}
