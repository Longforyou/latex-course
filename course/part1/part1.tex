\documentclass[10pt,times]{beamer}
\usepackage{amsfonts}
\usepackage{amsmath}
\usepackage{amssymb}
\usepackage{mathptmx}

\usepackage{color}
\usepackage{minted}
\usepackage{hyperref}
\usepackage{multicol}
\usepackage{tabularx}
\usepackage{booktabs}
\usepackage{menukeys}

% Stolen from John Miller's LaTeX course
\newcommand{\bftt}[1]{\textbf{\texttt{#1}}}
\newcommand{\comment}[1]{{\color[HTML]{008080}\textit{\textbf{\texttt{#1}}}}}
\newcommand{\cmmd}[1]{{\color[HTML]{008000}\bftt{#1}}}
\newcommand{\bs}{$\backslash$}
\newcommand{\cmdbs}[1]{\cmmd{\bs#1}}
\newcommand{\lb}{{\char'173}}% Left brackets -> {
\newcommand{\rb}{{\char'175}}% Right brackets -> }
\newcommand{\cmdbegin}[1]{\cmdbs{begin\lb}\bftt{#1}\cmmd{\rb}}
\newcommand{\cmdend}[1]{\cmdbs{end\lb}\bftt{#1}\cmmd{\rb}}



% this is where the example source files are loaded from
% do not include a trailing slash
\newcommand{\wllogo}{\textbf{Overleaf}}
\newcommand{\fileuri}{https://raw.githubusercontent.com/kks32/latex-course/master/exercises/}
\newcommand{\wlserver}{https://www.overleaf.com}
\newcommand{\wlnewdoc}[1]{\wlserver/docs?snip\_uri=\fileuri#1\&splash=none}

\def\tikzname{Ti\emph{k}Z}


% stolen from minted.dtx
\newenvironment{exampletwoup}
  {\VerbatimEnvironment
   \begin{VerbatimOut}{example.out}}
  {\end{VerbatimOut}
   \setlength{\parindent}{0pt}
   \fbox{\begin{tabular}{l| l}
   \begin{minipage}{0.55\linewidth}
     \inputminted[fontsize=\small,resetmargins]{latex}{example.out}
   \end{minipage} &
   \begin{minipage}{0.35\linewidth}
     \input{example.out}
   \end{minipage}
   \end{tabular}}}

\newenvironment{exampletwouptiny}
  {\VerbatimEnvironment
   \begin{VerbatimOut}{example.out}}
  {\end{VerbatimOut}
   \setlength{\parindent}{0pt}
   \fbox{\begin{tabular}{l|l}
   \begin{minipage}{0.55\linewidth}
     \inputminted[fontsize=\scriptsize,resetmargins]{latex}{example.out}
   \end{minipage} &
   \begin{minipage}{0.35\linewidth}
     \setlength{\parskip}{6pt plus 1pt minus 1pt}%
     \raggedright\scriptsize\input{example.out}
   \end{minipage}
   \end{tabular}}}

% ******************************** Meta-data ***********************************
\mode<presentation>
{
  \usetheme{Madrid}
  \setbeamercovered{transparent}
}


\usepackage{caption}
\captionsetup{font=scriptsize, labelfont=scriptsize, justification=centering}

\title{Writing papers and thesis using \LaTeX2e}

\author {Krishna Kumar \inst{*}\thanks{kks32@cam.ac.uk} }

\institute[ University of Cambridge ] % (optional, but mostly needed)
{
  \inst{1}%
  King's College\\
  University of Cambridge
}

\date[LaTeX course 2014] % (optional, should be abbreviation of conference name)
{\LaTeX for Beginners}


% Delete this, if you do not want the table of contents to pop up at
% the beginning of each subsection:
%\AtBeginSubsection[]
%{
%  \begin{frame}<beamer>{Outline}
%    \tableofcontents[currentsection,currentsubsection]
%  \end{frame}
%}


% If you wish to uncover everything in a step-wise fashion, uncomment
% the following command: 

% \beamerdefaultoverlayspecification{<+->}

\subtitle{Part I: Writing papers using \LaTeX}
%***************************** Title page **************************************
\begin{document}
\begin{frame}
  \titlepage
\end{frame}
%*******************************************************************************
%**************************** Introduction *************************************
%*******************************************************************************
\section{Sections and subsections}

%*******************************************************************************
%******************************* Frame *****************************************
%*******************************************************************************
\begin{frame}{Sections}
\begin{itemize}
\item To generate sections in \LaTeX: \cmdbs{section\{name-here\}}
\item Subsection: \cmdbs{subsection\{name-here\}}
\item Subsubsection: \cmdbs{subsubsection\{name-here\}}
\item Subsections without numbering \cmdbs{subsection*\{name-here\}}

\end{itemize}
\end{frame}

%*******************************************************************************
%******************************* Frame *****************************************
%*******************************************************************************
\begin{frame}{Title and author name}
Before you begin typing your document, i.e., \cmdbs{begin\{document\}} you need 
to define the author name and title.
\begin{itemize}
\item Title of the document in \LaTeX: \cmdbs{title\{name-here\}}
\item Author name: \cmdbs{author\{name-here\}}
\item Set a specific date: \cmdbs{date\{date-here\}}
\item How do you not print date: \cmdbs{date\{\}}
\end{itemize}
\centering
This only defines what the title of the document, author name and date create. 
It does not print it. To print the meta-data, do \cmdbs{maketitle} after begin 
document
\end{frame}


\begin{frame}{$\backslash$documentclass$[$\textbf{options}$]$\{\}}
\begin{table}
\begin{tabularx}{0.9\textwidth}{p{0.2\textwidth} X}
\toprule
Options & What they do \\ \midrule
Xpt & Sets the size of the main font in the document. Default: 10pt. \\
a4paper, \newline letterpaper & Defines the paper size. Default: letter/A4. \\
fleqn & displays formulas left-aligned instead of centered. \\
leqno & Places the numbering of formulas on the left hand side instead of the right. \\
titlepage,\newline  notitlepage & Specifies whether a new page should be started after 
the document title or not. The article class does not start a new page by default, while 
report and book do.\\
onecolumn,\newline  twocolumn & Instructs LaTeX to typeset the document in one column or 
two columns. \\
\end{tabularx}
\end{table}
\end{frame}


%**********************************************FRAME***************************************

\begin{frame}{$\backslash$documentclass$[$\textbf{options}$]$\{\} cont \dots}
\begin{table}
\begin{tabularx}{0.9\textwidth}{p{0.15\textwidth} X}
\toprule
Options & What they do \\ \midrule
twoside,\newline  oneside & double or single sided output. Article and report are single 
sided and the book is double sided by default. \\
landscape & Changes the layout of the document to print in landscape mode. \\
openright,\newline  openany & Makes chapters begin either only on right hand pages or on 
the next page available. This does not work with the article class, as it does not know 
about chapters. \\
draft & Draft - no images. \\ \bottomrule
\end{tabularx}
\end{table}
\end{frame}

%*******************************************************************************
%******************************* Frame *****************************************
%*******************************************************************************
\begin{frame}{Fonts}
\begin{itemize}
\item \tiny $\backslash$tiny
\item \scriptsize $\backslash$scriptsize
\item \footnotesize $\backslash$footnotesize
\item \small $\backslash$small
\item \normalsize $\backslash$normalsize
\item \large $\backslash$large
\item \Large $\backslash$Large
\item \LARGE $\backslash$LARGE
\item \huge $\backslash$huge
\item \Huge $\backslash$Huge
\end{itemize}
\end{frame}

%*******************************************************************************
%******************************* Frame *****************************************
%*******************************************************************************
\begin{frame}{Exercise 4: Sections}
\begin{itemize}
\item Add title, author and print date
\item Set font size to 11 pt
\item Create sections and subsections
\end{itemize}
\begin{center}
\fbox{\href{\wlnewdoc{Ex4-Sections/Ex4-Sections.tex}}{%
Click to open this exercise in \wllogo{}}}
\end{center}

\begin{itemize}
\item Hint: Don't forget to do \cmdbs{maketitle} and don't forget 
\cmmd{begin\{document\}} and \cmmd{end\{document\}}
\fbox{\href{\wlnewdoc{Ex4-Sections/Ex4-Sections-solution.tex}}{%
click here to see my solution}}.
\end{itemize}

\end{frame}


%*******************************************************************************
%******************************* Frame *****************************************
%*******************************************************************************
\begin{frame}[fragile]{Exercise 4: Sections}
\begin{itemize}
\item Why are dollar signs \keys{\$} special? We use them to mark mathematics 
in text.\\[1ex]
\begin{exampletwouptiny}
% not so good:
Let a and b be distinct positive
integers, and let c = a - b + 1.

% much better:
Let $a$ and $b$ be distinct positive
integers, and let $c = a - b + 1$.
\end{exampletwouptiny}
\item Always use dollar signs in pairs --- one to begin the mathematics, and one
to end it.
\item \LaTeX{} handles spacing automatically; it ignores your spaces.
\begin{exampletwouptiny}
Let $y=mx+b$ be \ldots

Let $y = m x + b$ be \ldots
\end{exampletwouptiny}

\end{itemize}
\end{frame}

\end{document}
