% Exercise 3: In this exercise, we are going to fix the error messages and look in to warnings.

% You will encounter 3 different error messages and a warning!

% 1. Error on line 16, always check the previous and the subsequent lines for 
%error: artic1e.cls is not found, LaTeX cannot find the class called artic1e, 
%meaning it's not able to load the template for artic1e, check the spelling of 
%artic1e should be "article". 

% 2. The second error message mentions error on line 19: Too many }'s remove 
%the extra } this will fix the error. 

% 3. !Undefined control sequence error (line 21). LaTeX is complaining that 
%there is no such command. Check the spelling (case-sensitive)


\documentclass{artic1e} % change to article
\begin{document}

\section{errors}} % remove the extra }

I like \LaTeX. If only it \meph{liked} me.

% Warnings 
\section{Warnings}

An overfull hbox means that there is a hyphenation or justification problem: moving the last word on the line to the next line would make the spaces in the line wider than the current limit; keeping the word on the line would make the spaces smaller than the current limit, so the word is left on the line, but with the minimum allowed space between words, and which makes the line go over the edge.
\\

\section{Underfull hbox warning see LogFile}
% Make sure you check the log file and fix this warning! An underfull vbox warning is caused by the line break on line 23 remove the line break. 

This is a warning that LaTeX cannot stretch the line wide enough to fit, without making the spacing bigger than its currently permitted maximum. he badness (0-10,000) indicates how severe this is (here the badness is severe 100000)It says what lines of your file it was typesetting when it found this, and the number in square brackets is the number of the page onto which the offending line was printed. The codes separated by slashes are the typeface and font style and size used in the line. Ignore them for the moment.

This comes up if you force a linebreak, e.g., $\backslash$$\backslash$, and have a return before it. Normally TeX ignores linebreaks, providing full paragraphs to ragged text. In this case it is necessary to pull the linebreak up one line to the end of the previous sentence.

\end{document}
